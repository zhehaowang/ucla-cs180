% Author: Zhehao Wang 404380075 zhehao@cs.ucla.edu

% Thanks to Haitao Zhang for helping with (trying to) catch up with the class, and with the latex template
% Grammar package: http://tex.stackexchange.com/questions/24886/which-package-can-be-used-to-write-bnf-grammars

\documentclass{article}
\topmargin = 0in
\oddsidemargin = 0in
\evensidemargin = \oddsidemargin
\textwidth = 6.5in
\textheight = 8in
\usepackage{bcprules}
\usepackage{amsthm}
\usepackage{amsmath}
\usepackage{syntax}

\newcommand{\step}[2]{{\tt #1} $\longrightarrow$ {\tt #2}}
\newcommand{\eval}[2]{{\tt #1} $\Downarrow$ {\tt #2}}
\newcommand{\tc}[3]{{\tt #1} $\vdash$ {\tt #2} \ : \ {\tt #3}}
\newcommand{\tcDef}[2]{{\tt #1}\ : \ {\tt #2}}

\newcommand{\inferrule}[3]{\infrule[#1]{\mbox{#2}}{\mbox{#3}}}
\newcommand{\inferax}[2]{\infrule[#1]{\mbox{}}{\mbox{#2}}}

% To work with <> symbols in {grammar}, as the stackexchange link pointed out.
\renewcommand{\syntleft}{\normalfont\itshape}
\renewcommand{\syntright}{}

\title{Homework 1}
\author{Zhehao Wang 404380075 (Dis 1D)}
\date{Apr 4, 2016}

\begin{document}
\maketitle

\begin{description}

\item[1]{Binary addition algorithm correctness proof}
  
  The input number $n$ can be denoted as $n=a_k...a_0$ in binary, where $a_i=0$, or $a_i=1$ ($0$ $\leq$ $i$ $\leq$ $k$, $k$ is the most significant bit). The flipped number $n'$ can be denoted as  $n'=a'_k...a'_0$ in binary.
  We have 
  \[
  n = \sum_{j=0}^{k}{a_j \cdot 2^j}
  \qquad \text{and} \qquad 
  n' = \sum_{j=0}^{k}{a'_j \cdot 2^j}
  \]
  
  Denote the position of first $0$ in $n$ from right to left to $i$, we have 
  $$n = \sum_{j=0}^{i-1}{1 \cdot 2^j} + 0 \cdot 2^i + \sum_{j=i+1}^{k}{a_j \cdot 2^j} = \frac{2^i-1}{2-1} + \sum_{j=i+1}^{k}{a_j \cdot 2^j} = 2^i - 1 + \sum_{j=i+1}^{k}{a_j \cdot 2^j}$$ 
  and
  $$n + 1 = 2^i + \sum_{j=i+1}^{k}{a_j \cdot 2^j}$$

  After the flip in question, resulting number $n'$ can be denoted as
  $$n' = \sum_{j=0}^{k}{a'_j \cdot 2^j} = \sum_{j=0}^{i-1}{0 \cdot 2^j} + 1 \cdot 2^i + \sum_{j=i+1}^{k}{a'_j \cdot 2^j} = 2^i + \sum_{j=i+1}^{k}{a_j \cdot 2^j}$$

  Thus we have $n' = n + 1$, and the binary addition algorithm in question is correct.

\item[2]{Binary tree depth algorithm}

 

\item[3]{Elementary-school-division algorithm}

\item[4]{NIM game}

(a)

(b)

\end{description}

\end{document}

int maxDepth(Node node) {
    if (node == NULL) {
        return (0);
    } else {
        // compute the depth of each subtree
        int lDepth = maxDepth(node.left);
        int rDepth = maxDepth(node.right);
        // use the larger one
        if (lDepth > rDepth)
            return (lDepth + 1);
        else
            return (rDepth + 1);
    }
}

O(n) with the number of nodes in the tree, as each node will be visited exactly once